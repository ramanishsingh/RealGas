%! Author = rfde2
%! Date = 4/8/2020

% Preamble
\documentclass[11pt]{article}
\author{Robert F. DeJaco}
\usepackage{xcolor}

%\makeatletter
%\newcommand*{\citenumns}[2][]{%
%    \begingroup
%    \let\NAT@mbox=\mbox
%    \let\@cite\NAT@citenum
%    \let\NAT@space\NAT@spacechar
%    \let\NAT@super@kern\relax
%    \renewcommand\NAT@open{}%
%    \renewcommand\NAT@close{}%
%    \cite[#1]{#2}%
%    \endgroup
%}
%\makeatother

%\usepackage{nomencl}    % for nomenclature section
%\usepackage{etoolbox}
%\usepackage{ifthen}
%\renewcommand\nomgroup[1]{%
%    \ifthenelse{\equal{#1}{A}}{%
%        \item[\textbf{Acronyms and Abbreviations}]}{%                A - Acronyms
%        \ifthenelse{\equal{#1}{R}}{%
%            \item[\textbf{Roman Symbols}]}{%                         R - Roman
%            \ifthenelse{\equal{#1}{G}}{%
%                \item[\textbf{Greek Symbols}]}{%                     G - Greek
%                \ifthenelse{\equal{#1}{S}}{%
%                    \item[\textbf{Superscripts}]}{%                  S - Superscripts
%                    \ifthenelse{\equal{#1}{U}}{%
%                        \item[\textbf{Subscripts}]}{%                U - Subscripts
%                        \ifthenelse{\equal{#1}{X}}{%
%                            \item[\textbf{Other Symbols}]}{%         X - Other Symbols
%                            {}}}}}}}}
% -----------------------------------------

%\makenomenclature
\newcommand{\ee}{\epsilon_{\mathrm{e}}}
\newcommand{\ep}{\epsilon_{\mathrm{p}}}
\newcommand{\rs}{\rho_{\mathrm{s}}}
\newcommand{\textn}{\text{n}}
\newcommand{\textm}{\text{m}}
\newcommand{\textw}{\text{w}}
\newcommand{\textref}{\text{ref}}
\newcommand{\textf}{\text{f}}
\newcommand{\ctm}{c_{\text{t,m}}}
\newcommand{\ctf}{c_{\text{t,f}}}
\newcommand{\cif}{c_{\text{i,f}}}
\newcommand{\Diax}{D_{i,\text{ax}}}

%\newcommand{\texta}{\texta}
%\nomenclature[U]{a}{Adsorbate}
%\nomenclature[U]{n}{Nanopore}

% Packages
\usepackage{amsmath}

\title{Derivation of Energy Balance}

% Document
\begin{document}

    \maketitle

    \subsubsection{General Form}

    If we neglect the following contributions to the energy balance:
    \begin{enumerate}
        \item Kinetic and potential energies
        \item Viscous dissipation of energy
        \item Radiation
        \item Variations in particle ($\ep$) and external ($\ee$) void fractions
    \end{enumerate}

    and no chemical reactions except adsorption occur, the
    energy balance on the nanoporous domain is expressed as~\cite{Walton2003}

    \begin{equation}
        \rs\frac{\partial U_\textn}{\partial t} = - \nabla Q_\textn
    \end{equation}

    where $\rho_{\mathrm{s}}$ is the structural or crystalline density of solid (the density of crushed or compressed solid containing no pores) [kg/m~$^3$],
    $U_\textn$ is the internal energy within the nanopore [J/kg],
    and $Q_\textn$ is the energy flux within the nanopore [W/m~$^2$].

    The energy balance on the macropore space is

    \begin{equation}
        \ep\frac{\partial (\ctm U_\textm)}{\partial t} + (1-\ep)\rs\frac{\partial \overline{U_\mathrm{n}}}{\partial t} = - \ep\nabla Q_\textm
    \end{equation}

    where $\overline{U_{\textn}}$ is the average internal energy of the nanoporous domain [J/kg]
    including the interface within the macropores (the surface touching the macropores),
    $U_\textm$ is the internal energy within the macropores [J/mol],
    and $\ep$ is the particle void fraction.
    The energy flux $Q_\textm$ includes contributions within the macropore and
    the interface between the macropore and nanopore.

    The energy balance on the interstitial fluid is

    \begin{equation}
        \ee\frac{\partial (\ctf U_\textf)}{\partial t} + \ee\nabla( u\ctf U_\textf) +
            \ep(1-\ee)\frac{\partial (\widehat{\ctm U_\textm})}{\partial t} + (1-\ee)(1-\ep)\rs\frac{\partial \hat{U_\mathrm{n}}}{\partial t} = - \ee\nabla Q_\textf
    \end{equation}

    where $\widehat{\ctm U_\textm}$ and $\hat{U_\textn}$ are values
    averaged over the space within the macropores,
    $U_\textf$ is the internal energy of the interstitial fluid [J/mol],
    $\ee$ is the external void fraction,
    and $Q_\textf$ is the energy flux within the feed.

    \subsubsection{Simplifications}
    We make the following assumptions

    \begin{enumerate}
        \item Each thermodynamic property $X(z,\theta,r,t)$ can be approximated as its corresponding average value along $\theta$ and $r$,
        so that $X=X(z,t)$
        \item The rate equation for energy transfer at the interface of the particle and bulk fluid can be written using a convective heat-transfer coefficient
        \item The heat loss at the wall can be approximated using a convective heat transfer coefficient
    \end{enumerate}

    The energy balance within the macropore becomes

    \begin{equation}
        \ep\frac{\partial (\ctm U_\textm)}{\partial t} + (1-\ep)\rs\frac{\partial U_\mathrm{n}}{\partial t} = - h_\textm a_{\textm} (T_\textm - T_\textf)
    \end{equation}

    The energy balance on the interstitial becomes

    \begin{equation}
        \ee\frac{\partial (\ctf U_\textf)}{\partial t} + \ee\frac{\partial ( u\ctf U_\textf)}{\partial z} -
        (1-\ee)h_\textm a_{\textm} (T_\textm - T_\textf)
        = - h_\textw a_{\textw} (T_\textf - T_\textw) - \ee\frac{\partial Q_\textf}{\partial z}
    \end{equation}

    \subsubsection{Simplification of the Interstitial Energy Balance}
    Neglecting radiation, the multicomponent energy flux in the interstitial fluid
    can be expressed as

    \begin{equation}
        Q_\textf = -K_\textf\frac{\partial T_\textf}{\partial z} + \sum_i \bar{H}_{i,\textf} D_{i,\text{ax}}\frac{\partial Y_{i,\textf}}{\partial z}
    \end{equation}

    where $\bar{H}_{i,\textf}$ is the partial molar enthalpy of component $i$ in the interstitial fluid.

    Using the additional definition $\ctf U_\textf=\sum_i\cif\bar{H}_{i,\textf}  - P\sum_i\cif\bar{V}_{i,\textf}$, the interstitial energy balance can be simplified to

    \begin{align}
        &\ee\frac{\partial (\sum_i\cif\bar{H}_{i,\textf} - P\sum_i\cif\bar{V}_{i,\textf})}{\partial t}
        + \ee\frac{\partial}{\partial z}\left(u\sum_i\cif\bar{H}_{i,\textf}\right)
        - \ee\frac{\partial (u P\sum_i\cif\bar{V}_{i,\textf})}{\partial z} -
        (1-\ee)h_\textm a_{\textm} (T_\textm - T_\textf)\nonumber\\
        &= - h_\textw a_{\textw} (T_\textf - T_\textw)
        + \ee\frac{\partial}{\partial z}\left(K_\textf\frac{\partial T_\textf}{\partial z}\right)
        + \ee\frac{\partial}{\partial z}\left(\sum_i \bar{H}_{i,\textf} D_{i,\text{ax}}\frac{\partial Y_{i,\textf}}{\partial z}\right)
    \end{align}

    Since $\bar{V}_{i,\textf}=V_{i,\textf}+\bar{V}_{i,\textf}^{\text{E}} $,
    and $\bar{H}_{i,\textf}=H_{i,\textf}+\bar{H}_{i,\textf}^{\text{E}}$ (where
    the superscript E denotes an excess property)
    the mass balance becomes

    \begin{align}
        &\ee\frac{\partial}{\partial t}\left(
            \sum_i\cif(\bar{H}_{i,\textf}^{\text{E}} - P\bar{V}_{i,\textf}^{\text{E}})
        \right)
        + \ee\frac{\partial}{\partial z}\left(
        u \sum_i \cif (\bar{H}_{i,\textf}^{\text{E}} - P\bar{V}_{i,\textf}^{\text{E}})
        \right)
        - \ee \frac{\partial}{\partial z}\left(\sum_i\bar{H}_{i,\textf}^{\text{E}}\Diax\frac{\partial Y_{i,\textf}}{\partial z}\right)
        \nonumber \\
        &+\ee\frac{\partial}{\partial t}\left(
            \sum_i\cif (H_{i,\textf} - PV_{i,\textf})
        \right)
        + \ee\frac{\partial}{\partial z}\left(
            u\sum_i \cif (H_{i,\textf}  - PV_{i,\textf})
        \right)
        - \ee \frac{\partial}{\partial z}\left(\sum_i H_{i,\textf}\Diax\frac{\partial Y_{i,\textf}}{\partial z}\right) \nonumber\\
        &= (1-\ee)h_{\textm}a_\textm(T_\textm - T_\textf) - h_\textw a_\textw(T_\textf - T_\textw) + \ee\frac{\partial}{\partial z}\left(K_\textf\frac{\partial T_{\textf}}{\partial z}\right)
    \end{align}

    For a pure component, the enthalpy can be calculated relative to some reference state at $P=P_\text{ref}$
    and $T=T_\text{ref}$ by constructing a thermodynamic path from $(P_\text{ref}, T_{\text{ref}})$
    to $(P_\text{ref}, T)$ to $(P, T)$.
    In this way, the enthalpy becomes

    \begin{equation}
        H_{i,\textf} = H_{i,\text{ref}} + \int_{T_{\textref}}^T\left(\frac{\partial H_i}{\partial T}\right)_P\mathrm{d}T
        + \int_{P_{\textref}}^P\left(\frac{\partial H_i}{\partial P}\right)_T\mathrm{d}P
    \end{equation}


    \bibliographystyle{unsrt}
    \bibliography{../docs/source/references}

\end{document}